% basic settings
\documentclass[a4paper,12pt]{article}
\usepackage[utf8]{inputenc}
\usepackage[czech]{babel}

% customizations
\usepackage{parskip} % paragraph indentation will be suppressed, instead there will be a space after
\usepackage[ddmmyyyy]{datetime} \renewcommand{\dateseparator}{.} % enables \today as dd.mm.yyyy
\usepackage{amsmath}
\usepackage{xcolor}
\usepackage{hyperref}
\usepackage[left = 2.0cm, right = 2.0cm, top = 2.5cm, bottom = 2.0cm]{geometry}

\begin{document}

\def\D{\mathrm{d}} % non-italic differential sign

\begin{center}
\LARGE\textbf{Protokol z projektu} \\
\vspace{2cm}
\large{
	Projekt na Numerické metody pro inženýry (P413003) \\
	Zadání č. 9 \\
	Vypracoval Ing. Jiří Zbytovský (\textcolor{blue}{\underline{\href{mailto:zbytovsi@vscht.cz}{email}}}) \\
	Odevzdáno \today
}
\end{center}

\newpage
\section*{Úloha 1}
Původní diferenciální rovnice popisující izotermní vnitřní difúzi v porézením katalyzátoru:
\begin{equation}
	\frac{\D^2 y}{\D x^2} + \frac{a}{x} \frac{\D y}{\D x} = \Phi^2 y^n
\end{equation}

S okrajovými podmínkami:
\begin{align}
	\frac{\D y}{\D x} (0) &= 0
	\\
	y(1) &= 1
\end{align}
Kde $y$~je bezrozměrná koncentrace, $x$~bezrozměrná souřadnice poloměru (0~je střed částice, 1~je její okraj), $a=1$~charakterizuje tvar částice (váleček), $\Phi=1$~Thieleho modul a $n=2$~řád reakce.

Diferenciální rovnice (DR) druhého řádu je třeba nahradit soustavou dvou DR prvního řádu, kde nezávislé proměnné jsou $y_1$ a $y_2$:
\begin{align}
	y_1 &= y
	\\
	y_2 &= y' = \D y / \D x
\end{align}

Výsledkem je tato soustava DR (s dosazením $\Phi$, $n$, $a$):
\begin{align}
	\frac{\D y_1}{\D x} = y_1' = f_1(x,y_1,y_2) &= y_2
	\\
	\frac{\D y_2}{\D x} = y_2' = f_2(x,y_1,y_2) &= -\frac{a}{x} y_2 + \Phi^2 y_1^n = y_1^2 - y_2 / x
\end{align}

S okrajovými podmínkami:
\begin{align}
	y_2(0) &= 0
	\\
	y_1(1) &= 1
\end{align}

Tento okrajový problém nyní řeším převedením na počáteční problém metodou střelby s variačními rovnicemi.
Volím proto $\eta$ jako odhad počáteční hodnoty $y_1(0) = \eta$ (zatímco $y_2(0)$ je dáno přesně).
Volba $\eta$ je přijatelná tehdy, když je přibližně splněna okrajová podmínka vyjádřená jako $\phi(\eta) = 0$, kde funkce $\phi$ vyjadřuje residuum rovnice okrajové podmínky při odhadu $\eta$:
\begin{equation}
	\phi(\eta) = y_1(1, \eta) - 1
\end{equation}

Aby byla přibližně splněna okrajová podmínka, použiji Newtonovu metodu na získání vhodnější hodnoty $\eta$:
\begin{equation}
	\eta_{k+1} = \eta_{k} + \frac{\phi(\eta_k)}{\phi'(\eta_k)}
\end{equation}

Kde $\phi'(\eta_k)$ je derivace $\phi$ podle $\eta$. Pro její vyčíslení zavedu variační rovnice pro nové nezávislé veličiny $p_1$, $p_2$:
\begin{align}
	p_1 &= \frac{\partial y_1}{\partial \eta}
	\\
	p_2 &= \frac{\partial y_2}{\partial \eta}
\end{align}

Jejich derivace lze vyjádřit řetízkovým pravidlem, čímž jsou získány další diferenciální rovnice do výše uvedené soustavy DR:
\begin{align}
	p_1' &=
	\frac{\partial p_1}{\partial x} =
	\frac{\partial^2 y_1}{\partial x \partial \eta} =
	\frac{\partial f_1}{\partial \eta} =
	\frac{\partial f_1 \partial y_1}{\partial y_1 \partial \eta} + \frac{\partial f_1 \partial y_2}{\partial y_2 \partial \eta} =
	\frac{\partial f_1}{\partial y_1} p_1 + \frac{\partial f_1}{\partial y_2} p_2 =
	p_2
	\\
	p_2' &=
	\frac{\partial p_2}{\partial x} =
	\frac{\partial^2 y_2}{\partial x \partial \eta} =
	\frac{\partial f_2}{\partial \eta} =
	\frac{\partial f_2 \partial y_1}{\partial y_1 \partial \eta} + \frac{\partial f_2 \partial y_2}{\partial y_2 \partial \eta} =
	\frac{\partial f_2}{\partial y_1} p_1 + \frac{\partial f_2}{\partial y_2} p_2 =
	2 y_1 p_1 - p_2 / x
\end{align}

S odpovídajícími počátečními podmínkami:
\begin{align}
	p_1(0) &= \frac{\partial y_1}{\partial \eta}(0) = 1
	\\
	p_2(0) &= \frac{\partial y_2}{\partial \eta}(0) = 0
\end{align}

Nyní lze vyjádřit derivaci $\phi'(\eta_k)$:
\begin{equation}
	\phi' = \frac{\partial \phi}{\eta} = \frac{\partial (y_1(1, \eta) - 1)}{\eta} = p_1(1, \eta)
\end{equation}

A tedy dosadit do vzorce pro Newtonovu metodu:
\begin{equation}
	\eta_{k+1} = \eta_{k} + \frac{y_1(1, \eta) - 1}{p_1(1, \eta)}
\end{equation}

Řeším tedy následující počáteční problém pomocí Runge-Kuttovy metody (použita implementace z open-source knihovny \textcolor{blue}{\underline{\href{https://scipy.org/}{scipy}}}).
\begin{align}
\label{ode1}
	y_1' &= y_2 &&
	y_1(0) = \eta_k
	\\
\label{ode2}
	y_2' &= y_1^2 - y_2 / x &&
	y_2(0) = 0
	\\
	p_1' &= p_2 &&
	p_1(0) = 1
	\\
	p_2' &= 2 y_1 p_1 - p_2 / x &&
	p_2(0) = 0
\end{align}

Iteraci $k$ ukončím tehdy, když $|\eta_{k+1} - \eta_{k}| < \epsilon$, kde $\epsilon$ je zvolená mez pro stanovení konvergence.
Následně řeším soustavu DR bez variačních rovnic, čili pouze rovnice~\ref{ode1}~a~\ref{ode2}. \\
Řešením soustavy jsou vektory $x$, $y_1$ a $y_2$, což je výsledek této úlohy.

Implementace tohoto postupu je v souboru \textit{app/uloha\_1.py}.

\newpage
\section*{Úloha 2}
asdf
\end{document}
